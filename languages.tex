\input{preamble.tmpl}

\title{Languages}
\author{ian.mcloughlin@gmit.ie}
\date{\today}

\begin{document}

\maketitle

\begin{abstract}
  A language is a subset of the Kleene star of an alphabet.
  An alphabet is just a set.
\end{abstract}

\section{Definitions}
  A set is a collection of objects and we call these objects elements of the set.
  Sets have no order and there is no repetition of elements.
  Every object is either in the set or not and every element is distinct.

  In the following notes we call the set we start with a finite set which we will call the alphabet, where by finite we mean the set has a fixed number of elements.
  A common set to use as an alphabet is:
  \[ A = \{ 0, 1 \} \]
  It's convenient if the elements of the alphabet represented as small, distinct symbols like \( 0 \) and \( 1 \).

  From an alphabet we generate finite ordered lists of elements that we call strings or words.
  We usually write these as elements of the alphabet side-by-side, and call these side-by-side elements characters.
  For example, \( 01011 \) is a string over \( A \) where the first and third characters are \( 0 \) and the second, fourth and fifth are \( 1 \).
  The string is said to have length five because there are five characters in it.
  A language is a set of strings over an alphabet.
  
\end{document}
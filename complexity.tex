% Short article template
\documentclass[dvipsnames, hidelinks]{IEEEtran}

% Enables the use of colour.
\usepackage{xcolor}
% Syntax high-lighting for code. Requires Python's pygments.
\usepackage{minted}
% Enables the use of umlauts and other accents.
\usepackage[utf8]{inputenc}
% Diagrams.
\usepackage{tikz}
% Settings for captions, such as sideways captions.
\usepackage{caption}
% Symbols for units, like degrees and ohms.
\usepackage{gensymb}
% Latin modern fonts - better looking than the defaults.
\usepackage{lmodern}
% Allows for columns spanning multiple rows in tables.
\usepackage{multirow}
% Better looking tables, including nicer borders.
\usepackage{booktabs}
% More math symbols.
\usepackage{amssymb}
% More math layouts, equation arrays, etc.
\usepackage{amsmath}
% More math fonts, like mathbb.
\usepackage{amsfonts}
% More theorem environments.
\usepackage{amsthm}
% More column formats for tables.
\usepackage{array}
% Adjust the sizes of box environments.
\usepackage{adjustbox}
% Better looking single quotes in verbatim and minted environments.
\usepackage{upquote}
% Better blank space decisions.
\usepackage{xspace}
% Better looking tikz trees.
\usepackage{forest}
% URLs.
\usepackage{hyperref}
% For plotting.
\usepackage{pgfplots}
% Images.
\usepackage{graphicx}
\usepackage{float}
\usepackage{setspace}
% Nice symbols for degrees, etc.
\usepackage{gensymb}
% Nicer tables.
\usepackage{tabularx}
% More nicer tables.
\usepackage{booktabs}
% For doing programmatical stuff.
\usepackage{arrayjob}
% Multiple columns.
\usepackage{multicol}
% Uncomment for watermarks.
% \usepackage{draftwatermark}

% For definitions.
\newtheorem*{definition}{Definition}

% Various tikz libraries.
% For drawing mind maps.
\usetikzlibrary{mindmap}
% For adding shadows.
\usetikzlibrary{shadows}
% Extra arrows tips.
\usetikzlibrary{arrows.meta}
% Old arrows.
\usetikzlibrary{arrows}
% Automata.
\usetikzlibrary{automata}
% For more positioning options.
\usetikzlibrary{positioning}
% Creating chains of nodes on a line.
\usetikzlibrary{chains}
% Fitting node to contain set of coordinates.
\usetikzlibrary{fit}
% Extra shapes for drawing.
\usetikzlibrary{shapes}
% For markings on paths.
\usetikzlibrary{decorations.markings}
% For advanced calculations.
\usetikzlibrary{calc}

% Make Tikz do new stuff.
\pgfplotsset{compat=1.16}

% GMIT colours.
\definecolor{gmitblue}{RGB}{20,134,225}
\definecolor{gmitred}{RGB}{220,20,60}
\definecolor{gmitgrey}{RGB}{67,67,67}

% Set some minted styling.
\usemintedstyle{bw}
\setminted{frame=lines,framesep=2mm,baselinestretch=1.2,fontsize=\footnotesize,linenos}

% Uncomment for a watermark for solutions.
% \SetWatermarkText{Solutions}

% Make bullet points looks nicer.
\renewcommand\labelitemi{--}

% Use courier for typewriter font.
\renewcommand{\ttdefault}{pcr}

\renewcommand{\familydefault}{\sfdefault}
% Increase line spacing.
% \onehalfspacing
 
% A center paragraph column type for tables.
\newcolumntype{x}[1]{>{\centering\arraybackslash\hspace{0pt}}p{#1}}

% Define the blank symbol for Turing machines.
\newcommand{\bl}{\( \sqcup \)}


\title{Computation}
\author{ian.mcloughlin@gmit.ie}
\date{\today}

\newcommand{\CC}[1]{\textbf{#1}}

\begin{document}

\maketitle

\begin{abstract}
  The following notes are about computation.
  We consider the limits and efficiency of computation.
  We are largely only interested in syntax.
\end{abstract}

\section{Languages}
Start with any set, call it an alphabet and denote it by \( A \).
  For example, take the set A = \{ 0, 1 \} is an alphabet.
  We define strings of length \( i \) over \( A \) recursively:
  \[
    A^0 = \{ \epsilon \}; \ 
    A^1 = A; \ 
    A^{i} = \{ as \mid a \in A, s \in A^{i-1} \}
  \]
  The \( \epsilon \) stands for the empty string\footnote{It's notoriously difficult to depict.}.

  The Kleene star \( A^* \) of the alphabet \( A \) is then union of all \( A^i \) for \( i \in \mathbb{N}_0 \).
  Here \( \mathbb{N}_0 \) is the set of natural numbers including zero: \( \mathbb{N}_0 = \{ 0, 1, 2, 3, \ldots \} \).
  \[ A^* = \bigcup_{i \in \mathbb{N}_0} A^i\]

  A language \( L \) over an alphabet \( A \) is a subset of \( A^* \): \( L \subseteq A^* \).
  Note that the strings in a language all have finite length.

\section{Turing machines}
  Turing machines encapsulate the concept of a computer.
  They are simple machines, consisting of a single read/write head and an infinite tape of cells, all but a finite number of which are blank.  
  The head is over a single cell of the tape and is in any one of a finite number of states at a given point in time.
  Turing machines perform one small step at a time, which involves reading the symbol in the current cell, overwriting it with a symbol, moving to the cell to the left or right, and changing state.

  We define a (deterministic\footnote{All Turing machines will be assumed to be deterministic until we get to section \ref{sect:nondet}.}) Turing machine \( M \) by a 7-tuple:
  \[ M = ( Q, \Sigma, \Gamma, \delta, q_0, q_a, q_f ) \]
  where:
  \( Q \) is a set of states;
  \( \Sigma \) is the input alphabet, not containing the blank symbol \( \sqcup \);
  \( \Gamma \) is the tape alphabet, a superset of \( \Sigma \) containing \( \sqcup \);
  \( \delta \) is a map:
  \[ \delta: (Q \setminus \{ q_a, q_f \}) \times \Gamma \rightarrow \Gamma \times \{ L, R \} \times Q; \]
  \( q_0 \) is the initial states;
  \( q_a \) is the accept state, which is a terminal state;
  \( q_f \) is the reject state, also a terminal state.  
  
  Before the Turing machines begins its operation, a finite number of consecutive non-blank cells contain symbols and the head is over the left-most of these.
  These non-empty cells form the Turing machine's input and can be viewed as a string.

  When the Turing machine operates on an input, there are three possibilities.
  The first two are similar: the machine can accept the input by ending in the accept state or the machine can reject the input by ending in the reject state.
  In these cases the machine stops upon entering these states, which is usually called halting.
  The third possibility is that the machine might never stop operating.
  
  The subset of \( \Sigma^* \) (the Kleene star of the input alphabet) containing all elements that are accepted by the Turing machine \( M \) is called the language of the Turing machine \( L(M) \):
  \[ L(M) \subseteq \Sigma^* \]
  A Turing machine that halts on all inputs is called a decider and is said to decide its language.
  Note it's possible for two distinct Turing machines \( M_1 \) and \( M_2 \) to accept or decide the same language: \( L( M_1 ) = L( M_2 ) \).

\section{State tables}
  A Turing machine can be summarised in five-columned table called a state table.
  Each row of the table describes how to perform the transition function \( \delta \) for a given state and tape symbol.
  Table \ref{table:statetable} gives an example of a state table for a Turing machine that accepts all strings over the alphabet \( \{0, 1\} \) that contain an even number of \(1\)'s, including zero as an even number. 
  \begin{table}[H]
    \centering
    \begin{tabular}{x{1cm}x{1cm}x{1cm}x{1cm}x{1cm}}
      \toprule
      State & Input & Write & Move & Next \\
      \midrule
      \(q_0\) &   0 &   0 & R & \(q_0\) \\
      \(q_0\) &   1 &   1 & R & \(q_1\) \\
      \(q_0\) & \bl & \bl & L & \(q_a\) \\
      \midrule
      \(q_1\) &   0 &   0 & L & \(q_1\) \\
      \(q_1\) &   1 &   1 & L & \(q_0\) \\
      \(q_1\) & \bl & \bl & R & \(q_f\) \\
      \bottomrule
    \end{tabular}
    \caption{Turing machine state table}
    \label{table:statetable}
  \end{table}
  Provided we adopt some conventions, the state table can be used to describe the whole Turing machine.
  First, the initial state is the first one listed in the table.
  Second, the tape alphabet is given by the distinct symbols in the second column.
  Finally, the only difference between the input and tape alphabets is the blank symbol.
  Note that there is single row for each combination of state and tape symbol.


\section{Decision problems}
  A decider makes a binary choice for a given input: accept or reject.
  The Turing machine \( M \) in this case performs a map:
  \[ f:\Sigma^* \rightarrow \{ q_a, q_f \} \]
  Sometimes this map is called an indicator function, as it indicates which elements of \( \Sigma^* \) are in \( M \).
  In general, a map from a set to any set with two elements is called a decision problem.
  So, a decider performs a decision problem.

\section{Complexity}
  A decider takes a finite number of steps before halting on a given input.
  We are typically interested in knowing the maximum number of steps \( f(n) \) the decider takes on all inputs of length \( n \).
  We use the number of steps as a proxy for time --- it's common to call the number of steps the \emph{time} the Turing machine takes.
  
  We can often express \( f(n) \) using a simple formula, such as \( f(n) = 2n + 2 \).
  We are interested in knowing how \( f(n) \) grows in relation to \( n \).
  For this, we use big-O notation:
  \begin{definition}
    A function \( f(n) \) is said to be \( O(g(n)) \) if there exist two positive integers \( c \) and \( n_0 \) such that \( cg(n) \geq f(n) \) for all \(n \geq n_0 \).
  \end{definition}
  In the previous example, \( f(n) = 2n + 2 \) is \( O(n) \) which can be seen by setting \( g(n) = n \), \( c = 3 \) and \( n_0 = 2 \) as per Fig. \ref{plot:2n2}.
  \begin{figure}[H]
  \begin{tikzpicture}
    \begin{axis}[xmin=0, xmax=6, ymin=0, ymax=20, axis lines = left]
    \addplot[smooth, domain=0:6] {2*\x + 2} node[below right, pos=0.8] (endofplotsquare) {\(2n+2\)} ;
    \addplot[smooth, domain=0:6] {3*\x}     node[above left,  pos=0.9] (endofplotsquare) {\(3n\)} ;
    \end{axis}
  \end{tikzpicture}
  \caption{\( 2n + 2 \) is \( O(n) \)}
  \label{plot:2n2}
  \end{figure}
  We usually define the complexity class \CC{NTIME}\((f(n)) \) as the set of all languages for which there is a decider that takes \( O(f(n) \) steps.

\section{\CC{P}}
  We denote by \CC{P} the set of languages that are each decidable by some Turing machine in \( O(n^k) \) steps for some \( n \in \mathbb{N} \) where \( \mathbb{N} \) is the set of natural numbers without zero, \( \{ 1, 2, 3, \ldots\} \).
  Using the \CC{NTIME}\((f(n)) \) definition above, \CC{P} is equivalently defined as the union of \CC{NTIME}\((n^k)\) for all \( k \in \mathbb{N} \).
  
  The \CC{P} is short for polynomial.
  Here, we define a polynomial is an expression in a variable \( n \) made up of constants (elements on \( \mathbb{N}\)), powers of \( n \) to a constant, addition, subtraction and multiplication.
  For example, \( n^5 + 2n + 4 \) is a polynomial whereas \( 2^n + n^2 + 5 \) is not because it is made up of something other than those elements listed above, namely \( 2^n \).



\section{Non-determinism}
  \label{sect:nondet}
  We consider now a modification to Turing machines: we allow the Turing machine to transition to zero or more states simultaneously from a given state when reading a given symbol.
  This requires a modification to the transition function as follows:
   \[ \delta: Q \times \Gamma \rightarrow \Gamma \times \{ L, R \} \times \mathcal{P}(Q) \]
   where \( \mathcal{P}(Q) \) is the powerset of \( Q \) --- the set of all subsets of \( Q \).
  From the state table point of view, the effect is that there may be zero, one, or more rows for each combination of state and tape symbol.
  In practice, we can view this as the Turing machine being able to fork, at will, into two or more branches of computation.
  Note that every deterministic Turing machine is basically a non-deterministic Turing machine\footnote{You do have to change the transition function to map to \( \{ q \} \) rather than \( q \), but that's a trivial change.}.

  We say an input is accepted by a non-deterministic Turing machine if any of its branches end in the accept state.
  It may be that other branches end in the fail state or compute forever.
  If all branches halt for all inputs, we say the non-deterministic Turing machine decides its language.

  How does non-determinism affect the computations Turing machines can perform?
  The bad news is that this doesn't give a Turing machine any extra power.
  Turing proved in his 1936 paper \emph{On computable numbers with an application to the encheidungsproblem} that some languages are not the decidable by any Turing machine.
  There are even languages that cannot be accepted, let alone be decided.
  Allowing non-determinism does not change the languages that can be allowed or decided.

  However, it is not known whether computations can be done more efficiently on non-deterministic Turing machines.
  Examples abound of efficient non-deterministic Turing machines that accept or decide languages for which no known efficient deterministic Turing machine is known.
  It is unknown whether such deterministic Turing machines exist.
  A special case of this question is the \CC{P} versus \CC{NP} problem.


\section{\CC{NP}}
  The \CC{NP} complexity class is the non-deterministic equivalent of the deterministic \CC{P} complexity class.
  Define \( NTIME(f(n)) \) as the set of languages decidable by a non-deterministic Turing machine in \( O(f(n)) \).
  Then, \CC{NP} is the set of all languages decidable by non-deterministic Turing machine in \( O(f(n)) \) steps.

  Given that non-deterministic Turing machines may fork into many paths, how should we count the steps they take?
  The common way to count this is to consider the longest path taken, irrespective of whether it accepts or rejects.
  Thus a non-deterministic Turing machine takes \(O(f(n))\) steps if the maximum number of steps in its longest path on an input of length \(n\) is \(O(f(n))\).

  Note, because every deterministic Turing machine is also a non-deterministic Turing machine, every language in \CC{P} is in \CC{NP}.
  So, \CC{P} \( \subseteq \) \CC{NP}.
  The \CC{P} versus \CC{NP} problem is to determine whether \CC{P} \( = \) \CC{NP} or \CC{P} \( \subset \) \CC{NP}.

\section{Church-Turing thesis}
  The ideas surrounding Turing machines arose from Turing's investigations into computability.
  Alonzo Church was Turing's PhD supervisor and developed an equivalent system called lambda calculus.
  By equivalent, we mean that any computation that can be done on a Turing machine can be done using lambda calculus and vice versa.
  Lambda calculus is the basis for functional programming languages, and the difference between functional and imperative programming languages is mirrored in that between lambda calculus and Turing machines.
  
  Turing and Church were working on ideas posed by G{\"o}del and Hilbert which concerned the foundations of mathematics.
  Hilbert asked if mathematics was algorithmic in the sense that true mathematical statements could be deduced by some procedure from axioms.
  G{\"o}del had shown that there must be some statements in mathematics that are true but cannot be proved.
  Church and Turing proved, while developing their computational systems, that their systems could not compute everything.
  What has become known as the Church-Turing thesis is the idea that what we as humans think of as computation is perfectly encapsulated by Turing machines and so the limits on computation are real --- they are not just features of Turing machines.

\section{Reductions}
  Ultimately the output of a Turing machine is a binary accept/reject decision.
  However, Turing machines have another type of output: what is left on the tape once the machine halts.
  We might care about this, for instance, if we are interested in running two Turing machines sequentially on the one input.
  The first Turing machine might manipulate its input in order that the second Turing machine receives that input altered in some way.

  For example, consider the language \CC{EVENPARITY} containing all strings over the alphabet \( \{ 0,1 \} \) that contain an even number of \(1\)'s, including zero as an even number.
  Thus, \CC{EVENPARITY} \(= \{\epsilon, 0, 00, 11, 110, 101, 011, \ldots\} \).



\section{\CC{COMPLETE}}

\section{Undecidable problems}

\end{document}
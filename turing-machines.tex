\input{preamble.tmpl}

\title{Turing machines}
\author{ian.mcloughlin@gmit.ie}
\date{\today}

\begin{document}

\maketitle

\begin{abstract}
  The Turing machine is a conceptual model of computation.
  Originally defined in Alan Turing's 1936 paper \emph{On computable numbers with an application to the entsheidungsproblem}.
\end{abstract}

\section{Definitions}
  A Turing machine is a 7-tuple \( ( Q, \Sigma, \Gamma, \delta, q_0, q_a, q_f ) \) where:
    \( Q \) is a set of states;
    \( \Sigma \) is the input alphabet, not containing the blank symbol \( \sqcup \);
    \( \Gamma \) is the tape alphabet, a superset of \( \Sigma \) containing \( \sqcup \);
    \( \delta \) is a map from \( Q \times \Gamma \) to \( \Gamma \times \{ L, R, S \} \times Q \);
    \( q_0 \) is the initial states;
    \( q_a \) is the accept state;
    \( q_f \) is the reject state.    
\end{document}